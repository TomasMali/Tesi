% !TEX encoding = UTF-8
% !TEX TS-program = pdflatex
% !TEX root = ../tesi.tex

%**************************************************************
\chapter{Introduzione}
\label{cap:introduzione}
%**************************************************************

In questo capitolo viene presentata brevemente l'azienda Sanmarco Informatica S.p.A. presso cui è stato svolto lo stage e la necessità che ha fatto nascere l'idea di questo stage.\\
Inoltre si presentano la struttura dei capitoli della tesi ed alcune norme tipografiche che verranno usate all’interno della stessa. \\




\noindent Esempio di utilizzo di un termine nel glossario \\
\gls{api}. \\

\noindent Esempio di citazione in linea \\
\cite{site:agile-manifesto}. \\

\noindent Esempio di citazione nel pie' di pagina \\
citazione\footcite{womak:lean-thinking} \\

%**************************************************************
\section{L'azienda}

Sanmarco Informatica nasce negli anni '80 come \textit{Software house\ped{G}} specializzata nello sviluppo di applicazioni gestionali per aziende manifatturiere ed è oggi una \textit{leading company} italiana nella progettazione e realizzazione di soluzioni a supporto della riorganizzazione di vari processi aziendali e professionali. L'ambizione e la volontà di rinnovarsi hanno permesso all'azienda di evolversi attraverso esperienze  e scelte imprenditoriali di successo, che individuano nella specializzazione del proprio capitale umano l'elemento centrale. L'azienda, partner di \textit{IBM Italia\ped{G}}, cresce grazie all'impegno di 320 persone fra dipendenti e collaboratori, 13 distributori e 4 sedi: Grisignano di Zocco (VI) come sede principale e Reggio Emilia (RE), Tavagnacco (UD) e Vimercate (MB) come filiali.
Sanmarco Informatica è la prima ed unica azienda italiana entrata a far parte dell'\textit{Open Power Foundation IBM\ped{G}} e a gennaio 2016 ha ricevuto il riconoscimento internazionale \textit{Beacon Award} come finalisti a livello mondiale fra le aziende d'eccellenza che propongono soluzioni tecnologiche innovative in combinazione con il sistema \textit{Power\ped{G}} di IBM.




%**************************************************************
\section{L'idea}

L'idea di base del progetto di stage si basa sulla necessità di alcune aziende di gestire in maniera più efficiente ed immediata i loro ordini giornalieri, la disponibilità degli articoli, gli scadenzari, gli incassi eccettera. Questo acquisisce ancora maggiore importanza laddove l'azienda in questione disponga di un numero elevato di rappresentanti ed ognuno di questi dovrà gestire i punti descriti sopra in base al proprio ruolo aziendale. \\ \\

NextBi offre attualmente una soluzione a questo problema interrogando il database  gestionale e sucessivamente portando i risultati su una pagina web. Questa soluzione tradizionale però, seppure efficiente?, a causa dell'elevato numero di utenti porta i suoi limiti nella gestione di autenticazione , essendo infatti la pagine web scomodamente accessibile da uno smartphone. Nel ottimizzare questo aspetto sorge la necessità di studiare una soluzione nella quale tutta la parte di autenticazione per categoria di utenti venga gestita in maniera molto più dinamica e a tutti gli utenti a portata di mano senza il bisogno quindi di accedere a pagine web, ovviando cosi la necessità di autenticazione ripetuta.

%**************************************************************
\section{Organizzazione del testo}

\begin{description}
    \item[{\hyperref[cap:processi-metodologie]{Il secondo capitolo}}] descrive con maggiore attenzione i dettagli dello stage. Specifica con più precisione la necessità a cui rispondere, i requisiti e gli obiettivi previsti, le tecnologie usate e la pianificazione prevista;
    
    \item[{\hyperref[cap:descrizione-stage]{Il terzo capitolo}}] approfondisce  ...
    
    \item[{\hyperref[cap:analisi-requisiti]{Il quarto capitolo}}] approfondisce ...
    
    \item[{\hyperref[cap:progettazione-codifica]{Il quinto capitolo}}] approfondisce ...
    
    \item[{\hyperref[cap:verifica-validazione]{Il sesto capitolo}}] approfondisce ...
    
    \item[{\hyperref[cap:conclusioni]{Nel settimo capitolo}}] descrive ...
\end{description}

Riguardo la stesura del testo, relativamente al documento sono state adottate le seguenti convenzioni tipografiche:
\begin{itemize}
	\item gli acronimi, le abbreviazioni e i termini ambigui o di uso non comune menzionati vengono definiti nel glossario, situato alla fine del presente documento;
	\item per la prima occorrenza dei termini riportati nel glossario viene utilizzata la seguente nomenclatura: \emph{parola}\glsfirstoccur;
	\item i termini in lingua straniera o facenti parti del gergo tecnico sono evidenziati con il carattere \emph{corsivo}.
\end{itemize}