% !TEX encoding = UTF-8
% !TEX TS-program = pdflatex
% !TEX root = ../tesi.tex

%**************************************************************
\chapter{Descrizione dello stage}
\label{cap:descrizione-stage}
%**************************************************************

\intro{In questo capitolo viene descritto il modo più dettagliato il progetto dello stage, presentati gli obiettivi e le pianificazioni previste. infine vengono illustrate le tecnologie usate per la realizzazione del progetto. } \\

%**************************************************************
\section{Il progetto di stage}

L'azienda Sanmarco Informatica S.p.A. fornisce un servizio di BI per la gestione dei processi di budget. La soluzione attuale si basa su un applicativo web B2B. Essendo tale applicativo realizzato con tecnologie poco recenti si è mirato quindi a migliorarlo, aggiornando le tecnologie, mantenedo però la gran parte della struttura esistente invariata.  A questo scopo il team NextBI si è preso l'impegnativa di realizzare questo progetto chiamta appunto Budget. In base alla pianificazione, la dimensione di questo progetto risulta molto ampia (stimato il rilascio nell'ottobre 2019). Da questa stima si deduce infatti che l'interesse del progetto non è solo l'aggiornamento della  versione dell'applicativo ma anche integrando man mano nel tempo migliorie e servizi innovativi. Il mio progetto si inserisce quindi all'interno del progetto budget con lo scopo di realizzare un sistema portatile e flessibile per la gestione e il monitoraggio di alcuni processi aziendali come ad essempio: gli ordini degli articoli giornalieri (settimanali o mensili), la disponibilità, lo spedito, le scadenze, gli inevasi eccettera. 





\section{Obiettivi}

\section{Vincoli}

\section{Pianificazione}
\subsection{Tabella della pianificazione}

\section{Ambiebte di lavoro}
\subsection{Risorse hardware}
\subsection{Risorse software}
\subsection{Risorse informative}

\section{Tecnologie usate}