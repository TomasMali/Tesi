% !TEX encoding = UTF-8
% !TEX TS-program = pdflatex
% !TEX root = ../tesi.tex

%**************************************************************
\chapter{Descrizione dello stage}
\label{cap:descrizione-stage}
%**************************************************************

\intro{In questo capitolo viene descritto il modo più dettagliato il progetto dello stage, presentati gli obiettivi e le pianificazioni previste. infine vengono illustrate le tecnologie usate per la realizzazione del progetto. } \\

%**************************************************************
\section{Il progetto di stage}

L'azienda Sanmarco Informatica S.p.A. fornisce un servizio di BI per la gestione dei processi di budget. La soluzione attuale si basa su un applicativo web B2B. Essendo tale applicativo realizzato con tecnologie poco recenti si è mirato quindi a migliorarlo, aggiornando le tecnologie, mantenedo però la gran parte della struttura esistente invariata.  A questo scopo il team NextBI si è preso l'impegnativa di realizzare questo progetto, dandogli il nome, appunto, Budget. In base alla pianificazione, la dimensione di questo progetto risulta molto ampia (stimato il rilascio nell'ottobre 2019). Da questa stima si deduce infatti che l'interesse del progetto non si limita solo all'aggiornamento della versione dell'applicativo esistente, ma anche integrando man mano nel tempo migliorie e servizi innovativi. Il mio progetto si inserisce quindi all'interno del progetto budget con lo scopo di realizzare un sistema portatile e flessibile per la gestione e il monitoraggio di alcuni processi aziendali come sono ad esempio: gli ordini degli articoli giornalieri (settimanali o mensili), la disponibilità, lo spedito, le scadenze, gli inevasi eccettera. 
\\ \\
Il problema che sorge è quindi la mancanza di un sistema portatile e di facile installazione che interroghi il database gestionale, il quale ,in tempo reale fornisca un monitoraggio aggiornato dei dati d'interesse. Il tutto dovrà essere integrato facilmente nel progetto Budget che si sta sviluppando in parallelo. E' stato inoltre stabilito che il sistema dovesse anche tener conto della gestione di profilazione per utente. Lo scopo finale è quindi di avere un sistema robbusto e di facile utilizzo che diminuisca i costi di mantenimento degli sviluppatori. Questo perchè da un'analisi fatta dall'azienda stessa negli ultimi anni i costi di mantenimento dell'applicativo sono  stati significanti, questo è dovuto dal fatto che l'installazione e il funzionamento dell'applicativo preesistente doveva essere garantito su una larga gama di sistemi operativi e browser.
\\ \\
A questo si aggiunge l'opportunita di un miglioramento del programma preesistente per quanto riguarda  la gestine dell'inserimento e rimozione degli utenti con una profilazione dati per tipo utente.





\section{Obiettivi}

\section{Vincoli}

\section{Pianificazione}
\subsection{Tabella della pianificazione}

\section{Ambiebte di lavoro}
\subsection{Risorse hardware}
\subsection{Risorse software}
\subsection{Risorse informative}

\section{Tecnologie usate}