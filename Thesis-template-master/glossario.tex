
%**************************************************************
% Acronimi
%**************************************************************
\renewcommand{\acronymname}{Acronimi e abbreviazioni}

\newacronym[description={\glslink{apig}{Application Program Interface}}]
    {api}{API}{Application Program Interface}

\newacronym[description={\glslink{umlg}{Unified Modeling Language}}]
    {uml}{UML}{Unified Modeling Language}

%**************************************************************
% Glossario
%**************************************************************
%\renewcommand{\glossaryname}{Glossario}





\newglossaryentry{B2B}
{
    name=\glslink{B2B}{B2B},
    text=B2B,
    description={Acronimo di Business-to-Business, in italiano commercio interaziendale, sta ad indicare le transizioni elettroniche tra imprese.}
}

\newglossaryentry{Business Intelligence}
{
    name=\glslink{Business Intelligence}{BUSINESS INTELLIGENCE},
    text= Business Intelligence,
    description={Con questa locuzione si fa riferimento ad un insieme di processi aziendali per raccogleire dati ed analizzare informazioni strategiche, tecnologie utilizzate per realizzare questi processi e le informazioni ottenute come risultato di questi processi.}
}

\newglossaryentry{DAO}
{
    name=\glslink{DAO}{DAO},
    text= DAO,
    description={Acronimo di Data Acess Object, è un pattern architetturale per la gestione della persistenza: si tratta fondamentalmente di una classe con relativi metodi che rappresenta un'entitaà tabellare di un database relazionale, usat principalmente in applicazioni web. Il vantaggio relativo all'uso del DAO è dunque il mantenimento di una rigida separazione tra le componenti di un'applicazione.}
}
\newglossaryentry{data-layer}
{
    name=\glslink{data-layer}{Data-Layer},
    text= data-layer,
    description={E' uno strato di una applicazione che fornisce un accesso semplificato ai dati memorizzati nell'archiviazione persistente, come un database relazionale.}
}
\newglossaryentry{framework}
{
    name=\glslink{framework}{FRAMEWORK},
    text= framework,
    description={ Un framework, termine della lingua inglese che può essere tradotto come struttura, in informatica e specificatamente nello sviluppo software, è un'architettura logica di supporto (spesso un'implementazione logica di un particolare design pattern) su cui un software può essere progettato e realizzato, spesso facilitandone lo sviluppo da parte del programmatore.}
}
\newglossaryentry{IDE}
{
    name=\glslink{IDE}{IDE},
    text= IDE,
    description={In informatica un ambiente di sviluppo integrato è un software che, in fase di programmazione, aiuta i programmatori nello sviluppo del codice sorgente di un programma. Spesso l'IDE aiuta lo sviluppatore segnalando errori di sintassi del codice direttamente in fase di scrittura, oltre a tutta una serie di strumenti e funzionalità di supporto alla fase di sviluppo e debugging.}
}

\newglossaryentry{Open Power Foundation IBM}
{
    name=\glslink{Open Power Foundation IBM}{OPEN POWER FOUNDATION IBM},
    text= Open Power Foundation IBM,
    description={Open-Power Foundation è una collaborazione intorno ai prodotti di Power Architecture avviati da IBM e annunciata come Open-Power Consortium il 6 agosto 2013. IBM sta aprendo la tecnologia che circonda le offerte di Power Architecture come specifiche del processore, firmware e software e offre questo su una licenza free utilizzando un modello di sviluppo collaborativo con i loro partner.}
}




\newglossaryentry{open source}
{
    name=\glslink{open source}{OPEN SOURCE},
    text= open source,
    description={In informatica, il termine inglese open source (che significa sorgente aperta) indica un software di cui gli autori (più precisamente, i detentori dei diritti) rendono pubblico il codice sorgente, favorendone il libero studio e permettendo a programmatori indipendenti di apportarvi modifiche ed estensioni.}
}

\newglossaryentry{plug-in}
{
    name=\glslink{plug-in}{PLUG-IN},
    text= plug-in,
    description={Il plugin in campo informatico è un programma non autonomo che interagisce con un altro programma per ampliarne o estenderne le funzionalità originarie. Ad esempio, un plugin per un software di grafica permette l'utilizzo di nuove funzioni non presenti nel software principale}
}


\newglossaryentry{UML2.0}
{
    name=\glslink{UML2.0}{UML2.0},
    text= UML2.0,
    description={In ingegneria del software, UML (unified modeling language, "linguaggio di modellizzazione unificato") è un linguaggio di modellizzazione e specifica basato
sul paradigma orientato agli oggetti. La versione 2.0 è stata consolidata nel 2004 e ufficializzata nel 2005. UML 2.0 riorganizza molti degli elementi della versione precedente in un quadro di riferimento ampliato e introduce molti nuovi strumenti, inclusi alcuni nuovi tipi di diagrammi }
}



\newglossaryentry{Use Cases Diagram}
{
    name=\glslink{Use Cases Diagram}{USE CASES DIAGRAM},
    text= Use Cases Diagram,
    description={ In UML, gli Use Case Diagram (UCD o diagrammi dei casi d'uso) sono diagrammi dedicati alla descrizione delle funzioni o servizi offerti da un sistema, cosi come sono percepiti e utilizzati dagli attori che interagiscono col sistema stesso.}
}










