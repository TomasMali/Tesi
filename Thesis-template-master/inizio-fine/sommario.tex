% !TEX encoding = UTF-8
% !TEX TS-program = pdflatex
% !TEX root = ../tesi.tex

%**************************************************************
% Sommario
%**************************************************************
\cleardoublepage
\phantomsection
\pdfbookmark{Sommario}{Sommario}
\begingroup
\let\clearpage\relax
\let\cleardoublepage\relax
\let\cleardoublepage\relax

\chapter*{Sommario}

Il presente documento descrive il lavoro svolto durante il periodo di stage, della durata di circa trecento ore, dal laureando Tomas Mali presso l'azienda Sanmarco Informatica S.p.A.

L'obiettivo del lavoro è la progettazione e implementazione di una soluzione di Business Intelligence (BI) efficace ed efficiente  per la gestione di un insieme di processi aziendali  che riguardano la raccolta e l'analisi dati in modo rapido e sicuro, minimizzando i futuri costi di manutenzione dell'applicativo preesistente.\

Al fine di raggiungere l'obbiettivo di questo progetto, come primo passo, è stato necessario lo studio e la valutazione delle tecnologie innovative  per l'elaborazione e la manipolazione dei dati gestionali di grandi dimensioni, al fine di renderli di immediata disponibilità e portabilità. E' stato inoltre opportuno mostrarli, per quanto riguarda l'interazione con l'utente, attraverso una semplice e dinamica interfaccia mobile, affiancata da un'interfaccia web preesistente. Trattandosi dunque di raccolte dati provenienti da una fonte gestionale, risulta evidente l'accuratezza e l'attenzione  che bisogna prestare nel manipolare tali dati e presentarli, ad esempio sotto forma di documenti potabili come PDF, txt eccetera.\\ 
In secondo luogo inoltre è stato richiesto un miglioramento delle funzionalità dell'applicativo preesistente, in particolare un miglioramento nella gestione degli utenti (di vari ruoli come capoarea, superuser eccetera) i quali fanno uso dei processi  in modo profilato a seconda del tipo di utente.

%\vfill
%
%\selectlanguage{english}
%\pdfbookmark{Abstract}{Abstract}
%\chapter*{Abstract}
%
%\selectlanguage{italian}

\endgroup			

\vfill

