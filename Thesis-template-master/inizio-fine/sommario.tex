% !TEX encoding = UTF-8
% !TEX TS-program = pdflatex
% !TEX root = ../tesi.tex

%**************************************************************
% Sommario
%**************************************************************
\cleardoublepage
\phantomsection
\pdfbookmark{Sommario}{Sommario}
\begingroup
\let\clearpage\relax
\let\cleardoublepage\relax
\let\cleardoublepage\relax

\chapter*{Sommario}

Il presente documento descrive il lavoro svolto durante il periodo di stage, della durata di circa trecento ore, dal laureando Tomas Mali presso l'azienda Sanmarco Informatica S.p.A.\\ \\

Gli scopi sono stati la progettazione e la successiva implementazione di una soluzione BI (business intelligence) efficace ed efficiente per gestire un insieme di processi aziendali che riguardano la raccolta e l'analisi dati in modo rapido e sicuro, minimizzando i futuri costi di manutenzione dell’applicativo preesistente.\\\\

Gli obiettivi principali sono stati quindi lo studio e la valutazione delle tecnologie innovative all'avanguardia per l'elaborazione e la manipolazione di dati gestioneali di grandi dimensioni, al fine di renderli di immediata disponibilità e portabilità, presentandoli, per quanto riguarda l'interazione con l'utente, attraverso una semplice e dinamica interfaccia mobile, affiancata da un'interfaccia web preesistente. Trattandosi dunque di raccolte dati provenienti da una fonte gestionale, risulta evvidenta quindi l'attenzione e l'accuratezza  che bisogna prestare nel manipolare tali dati e presentarli, ad esempio sotto forma di documenti potatili come PDF, txt eccetera.\\ 
In secondo luogo inoltre è stato richiesto un miglioramento delle funzionalità dell'applicativo preesistente, in particolare un miglioramento nella gestione degli utenti (di vari ruoli come capoarea, superuser ecc..) i quali usano in modo profilato i processi a seconda del tipo di utente.

%\vfill
%
%\selectlanguage{english}
%\pdfbookmark{Abstract}{Abstract}
%\chapter*{Abstract}
%
%\selectlanguage{italian}

\endgroup			

\vfill

