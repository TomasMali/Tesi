% !TEX encoding = UTF-8
% !TEX TS-program = pdflatex
% !TEX root = ../tesi.tex

%**************************************************************
% Sommario
%**************************************************************
\cleardoublepage
\phantomsection
\pdfbookmark{Sommario}{Sommario}
\begingroup
\let\clearpage\relax
\let\cleardoublepage\relax
\let\cleardoublepage\relax

\chapter*{Sommario}

Il presente documento descrive il lavoro svolto durante il periodo di stage, della durata di circa trecento ore, dal laureando Tomas Mali presso l'azienda Sanmarco Informatica S.p.A.\\ \\
Gli scopi sono stati la progettazione e la sucecessiva implementazione di una soluzione efficace ed efficiente per gestire i processi di budget, di ordini e di disponibilità, in modo rapido e sempre a portata di mano, interrogando così il software gestionale già presente.\\ \\

Gli obiettivi principali sono stati quindi lo studio e la valutazione delle tecnologie mobile innovative all'avanguardia per l'elaborazione e la manipolazione di dati gestioneali di grandi dimensioni, al fine di renderli di immediata disponibilità e portabilità, presentandoli attraverso una semplice e dinamica interfaccia mobile affiancata da un'interfaccia web. Trattandosi dunque di raccolte dati provenienti da una fonte gestionale, risulta evvidenta per cui l'attenzione e l'accuratezza nel presentare tali dati sotto forma di documenti potatili (ad esempio PDF).
In secondo luogo inoltre è stato richiesto una gestione degli utenti (di vari ruoli come capoarea, superuser ecc..) destinati all'uso profilato dei processi a seconda del tipo di utente.

%\vfill
%
%\selectlanguage{english}
%\pdfbookmark{Abstract}{Abstract}
%\chapter*{Abstract}
%
%\selectlanguage{italian}

\endgroup			

\vfill

